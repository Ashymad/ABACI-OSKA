\documentclass[12pt]{oska}

\usepackage[english,polish]{babel}

\usepackage{polski}

\usepackage[utf8]{inputenc}
\usepackage[T1]{fontenc}
\usepackage{lmodern}  % zestaw fontów
\usepackage{indentfirst}
\frenchspacing

\usepackage[hidelinks]{hyperref}
\usepackage{enumerate}
\usepackage{graphicx}
\usepackage[all,defaultlines=2]{nowidow}
\usepackage{float}
\usepackage{makecell}
\usepackage{siunitx}
\sisetup{output-decimal-marker = {,}}
\usepackage{icomma}
\let\lll\undefined
\usepackage{amsmath, amssymb, amsfonts}
\usepackage{mathtools}
\usepackage{import}		% wklejanie pdf_tex
\usepackage{xcolor}		% kolory
\usepackage{microtype}
\usepackage{tabularx}
\usepackage{pgfplots}
\let\pgfimage\includegraphics

\usepackage{placeins}	% poprawia float

\let\Oldsection\section
\renewcommand{\section}{\FloatBarrier\Oldsection}

\let\Oldsubsection\subsection
\renewcommand{\subsection}{\FloatBarrier\Oldsubsection}

\let\Oldsubsubsection\subsubsection
\renewcommand{\subsubsection}{\FloatBarrier\Oldsubsubsection}

\setlength{\emergencystretch}{3em}

\usepackage[
style=numeric,
sorting=none,
language=autobib,
autolang=other,
backref=false,
isbn=true,
url=false,
maxbibnames=6,
minbibnames=6,
backend=biber
]{biblatex}

\AtBeginBibliography{
  \renewcommand\labelnamepunct{:\space}
  \renewcommand\newunitpunct{\addcomma\space}
  \renewcommand{\finentrypunct}{}

  \renewcommand{\bibopenparen}{\addcomma\addspace}
  \renewcommand{\bibcloseparen}{\addspace}
}

\usepackage{csquotes}
\DeclareQuoteAlias{croatian}{polish}

\usepackage{icomma}
\usepackage{newtxtext}

\usepackage{listings}
\lstloadlanguages{TeX}

\lstset{
  literate={ą}{{\k{a}}}1
  {ć}{{\'c}}1
  {ę}{{\k{e}}}1
  {ó}{{\'o}}1
  {ń}{{\'n}}1
  {ł}{{\l{}}}1
  {ś}{{\'s}}1
  {ź}{{\'z}}1
  {ż}{{\.z}}1
  {Ą}{{\k{A}}}1
  {Ć}{{\'C}}1
  {Ę}{{\k{E}}}1
  {Ó}{{\'O}}1
  {Ń}{{\'N}}1
  {Ł}{{\L{}}}1
  {Ś}{{\'S}}1
  {Ź}{{\'Z}}1
  {Ż}{{\.Z}}1,
  basicstyle=\footnotesize\ttfamily,
}

\usepackage{array}
\usepackage{tabularx}
\usepackage{multirow}
\usepackage{booktabs}
\usepackage{makecell}
\usepackage[flushleft]{threeparttable}

% defines the X column to use m (\parbox[c]) instead of p (`parbox[t]`)
\newcolumntype{C}[1]{>{\hsize=#1\hsize\centering\arraybackslash}X}

\setlength{\cftsecnumwidth}{10mm}
\setcounter{secnumdepth}{4}
\brokenpenalty=10000\relax

\titlePL{Zastosowanie uczenia maszynowego do wykrycia metody
kompresji stratnej sygnału audio}
\titleEN{Audio compression method identification through machine learning}
\affiliation{Akademia Górniczo-Hutnicza w Krakowie}

\namem{Szymon}
\surnamem{Mikulicz}
\email{szymon.mikulicz@posteo.net}

\namei{} \surnamei{}
\nameii{} \surnameii{}
\nameiii{} \surnameiii{}
\nameiiii{} \surnameiiii{}
\nameiiiii{} \surnameiiiii{}

\summaryPL{%
  W artykule opisano koncepcję stworzenia rozwiązania programowego
  wykorzystującego technologię uczenia maszynowego do rozpoznania metody
  kompresji stratnej zastosowanej w pliku audio. Aplikacja pozwoli
  zidentyfikować pliki w formacie bezstratnym (LPCM, FLAC itp.), które uzyskano
  poprzez dekompresję plików w formacie stratnym (MP3, AAC, OGG itp.).
  Zastosowanie uczenia maszynowego oraz zaawansowanych algorytmów ekstrakcji
  cech pozwoli wykryć metody kompresji, których nie da się wykryć przy pomocy
  algorytmów wyszukujących artefakty ani poprzez naoczne badanie
  charakterystyki częstotliwościowej.%
}

\summaryEN{%
  The paper describes an idea for a new software solution, that uses machine
  learning to identify method of lossy compression used in an audio file. The
  application will be able to identify files in a lossless format (LPCM, FLAC,
  etc.) that have been obtained by decompression of files in a lossy format
  (MP3, AAC, OGG, etc.). The usage of machine learning and advanced algorythms
  for feature extraction will allow to detect methods of compression that are
  not able to be detected by algorhythms searching for artifacts or by looking
  at the frequency spectrum.
}

\addbibresource{bibliografia.bib}

\begin{document}

\maketitles

\section{Wstęp}

Analogowy sygnał elektryczny reprezentowany jest w postaci cyfrowej najczęściej
jako LPCM (ang. \textit{Linear Pulse-Code Modulation}), na przykład w plikach
WAVE, jest to jednak format który jest bardzo mało efektywny pod względem
zajmowanej przestrzeni dyskowej, co ma szczególne znaczenie w przypadku
transmisji danych przez internet, gdzie dąży się do zmniejszenia rozmiarów
plików przesyłanych. W celu zmniejszenia ilości danych potrzebnych na zapis
sygnału audio stosuje się metody kompresji stratnej oraz bezstratnej. Częściej
stosowana w przesyle danych jest kompresja stratna, gdyż pozwala ona na
znacznie mniejszy stosunek rozmiaru pliku skompresowanego do oryginalnego na
poziomie nawet kilku procent, od stratnej, dla której ta wartość wynosi około
pięćdziesięciu procent. W artykule zostanie omówione w jaki sposób
przeprowadzana jest kompresja stratna i w jaki sposób można wykorzystać tę
wiedzę do ekstrakcji cech z sygnałów w celu trenowania modeli uczenia
maszynowego do rozróżnienia plików oryginalnych od plików zmodyfikowanych przez
algorytm kompresji stratnej.

\section{Kompresja stratna plików audio}

Kompresja stratna plików audio przebiega w kilku krokach. Najpierw wyznaczane
są współczynniki czasowo-częstotliwościowe sygnału, poprzez przeprowadzenie
transformacji. W większości metod stosuje się MDCT (ang. \textit{Modified
Digital Cosine Transform}), zespół filtrów PQMF (ang. \textit{Pseudo-Quadrature Filter
Banks}) lub rozwiązanie hybrydowe które wykorzystuje obie te transformacje.
Transformacja kosinusowa MDCT:
\begin{equation}\label{eq:mdct}
  X_k = \sum_{n=0}^{2N-1}x_n\cos\left[\frac{\pi}{N}\left(n+\frac{1}{2}+\frac{N}{2}\right)\left(k+\frac{1}{2}\right)\right],
\end{equation}
ma tę właściwość, że większość współczynników uzyskanych w jej wyniku ma bardzo
niewielkie wartości, co jest wykorzystywane w kompresji, gdyż w wyniku
kwantyzacji dużo współczynników osiąga wartość zero. Stopień kwantyzacji dla
fragmentów sygnału jest kontrolowany przez model psychoakustyczny, który jest
tworzony przez algorytm na podstawie sygnału wejściowego, co pozwala na
wycięcie informacji, które w minimalnym stopniu wpływają na odbiór sygnału
przez człowieka. Uproszczony schemat procesu kompresji stratnej przedstawia
rysunek \ref{fig:comp_scheme}.

\begin{figure}[!tbh]
  \centering
  \import{vecgraphics/}{comp_scheme.pdf_tex}
  \caption{Schemat kompresji stratnej plików audio}
  \label{fig:comp_scheme}
\end{figure}

Proces kompresji stratnej wprowadza do sygnału zniekształcenia, tak zwane
\textit{artefakty}, których występowanie wynika z zmiennych poziomów
kwantyzacji w wyniku działania modelu psychoakustycznego oraz z efektów
brzegowych powstałych w procesie transformacji i konsekwentnie transformacji
odwrotnej, podczas dekompresji sygnału.Rysunki od \ref{fig:mp3_320} do \ref{fig:mp3_128}
przedstawiają ten sam fragment muzyki po kompresji i dekompresji MP3 z coraz
mniejszą przepływnością (eng. \textit{bitrate}), wartością oznaczającą ilość
danych na sekundę sygnału. Na tych spektrogramach z łatwością można zauważyć
artefakty kompresji w szczególności jako wycięcia wysokich częstotliwości
zwiększające się wraz z malejącą przepływnością.
\begin{figure}[!tbh]
  \centering
  \import{plots/}{320.pgf}
  \caption{Spektrogram pliku MP3 \SI{320}{\kilo\bit\per\second}}
  \label{fig:mp3_320}
\end{figure}
\begin{figure}[!tbh]
  \centering
  \import{plots/}{256.pgf}
  \caption{Spektrogram pliku MP3 \SI{256}{\kilo\bit\per\second}}
  \label{fig:mp3_256}
\end{figure}
\begin{figure}[!tbh]
  \centering
  \import{plots/}{196.pgf}
  \caption{Spektrogram pliku MP3 \SI{196}{\kilo\bit\per\second}}
  \label{fig:mp3_196}
\end{figure}
\begin{figure}[!tbh]
  \centering
  \import{plots/}{128.pgf}
  \caption{Spektrogram pliku MP3 \SI{128}{\kilo\bit\per\second}}
  \label{fig:mp3_128}
\end{figure}
\section{Poprzednie prace}
Rozwiązania pozwalające wykryć rodzaj i parametry kompresji stratnej audio
zostały zaproponowane w poprzednich pracach w których także pokazano
ich skuteczność. W szczególności: wykorzystano dużą ilość parametrów
statystycznych fragmentu sygnału audio do konstrukcji wektorów cech
wykorzystanych następnie do trenowania algorytmów uczenia maszynowego%
\cite{Hicsonmez2011AudioCI}, następnie przetestowano otrzymaną sieć na serii
sygnałów mowy i muzyki skompresowanych różnymi kodekami z przepływościami poniżej
\SI{128}{\kilo\bit\per\second}, otrzymując średnią dokładność
\SI{94.77}{\percent}\cite{Hicsonmez2013MethodsIT}; wykorzystano sieci CNN (eng.
\textit{Convolutional Neural Network}), metodę uczenia maszynowego stosowaną
często w rozpoznawaniu obrazów, które trenowano na spektrogramach fragmentów
sygnałów audio (parametry spektrogramu ustalono odgórnie), otrzymując średnią
dokładność około \SI{98}{\percent} testując uzyskaną sieć na plikach
skompresowanych różnymi kodekami z przepływnością od \num{32} do
\SI{320}{\kilo\bit\per\second}\cite{Hennequin2017CodecIL}; skonstruowano także
algorytm niewykorzystujący uczenia maszynowego, który rozpoznaje metodę
kompresji audio na podstawie różnic energii pomiędzy kolejnymi transformatami o
parametrach stosowanych do kompresji danym kodekiem, przetestowany na muzyce i
fragmentach ścieżki filmowej uzyskując dokładność
\SI{99.6}{\percent}\cite{Kim2018LossyAC}.

Jak widać istnieje wiele metod które bardzo skutecznie wykrywają rodzaj
kodowania zastosowanego do pliku, szczególnie interesujące jest rozwiązanie
niewykorzystujące uczenia maszynowego, lecz posiada ono wady: nie pozwala w
obecnej postaci na wykrycie parametrów kompresji oraz, przede wszystkim, wymaga
znajomości parametrów stosowanych do kompresji danego kodeka audio, co może być
trudne w przypadku kodeków własnościowych których architektura jest niedostępna
publicznie.
\section{Nowa koncepcja}
Zaproponowano nowy system detekcji kompresji audio, który poszerza metodę
opisaną w \cite{Kim2018LossyAC}. Aby zredukować ograniczenia oryginalnego
rozwiązania, zamiast stosować metodę znajdywania maksimum i liczenia średniej
geometrycznej z uzyskanych różnic energii uzyskane transformaty wykorzystane
zostaną do treningu sieci neuronowej w celu wydobycia z nich informacji o
metodzie i poziomie kompresji sygnału. Ponadto zostanie wytrenowana druga sieć
która na podstawie spektrogramu sygnału wyznaczy rodzaj i parametry
transformaty zastosowanej do kompresji sygnału tak aby współczynnik pewności
(ang.  \textit{confidence}) sieci rozpoznającej był jak największy.  Pozwoli to
na skuteczne rozpoznanie rodzaju i parametrów kodeka zastosowanego do kompresji
pliku bez konieczności posiadania uprzedniej wiedzy o działaniu kodeków.
Schemat proponowanego rozwiązania przedstawia rysunek \ref{fig:new_scheme}.

\begin{figure}[!tbh]
  \centering
  \import{vecgraphics/}{new_scheme.pdf_tex}
  \caption{Schemat kompresji stratnej plików audio}
  \label{fig:new_scheme}
\end{figure}

\section{Wnioski}

Na podstawie rozwiązań które zostały opisane w literaturze zaproponowano nowy
algorytm do detekcji kompresji audio, którego celem jest pozbycie się
ograniczeń dotychczasowych algorytmów, jednakże nie przeprowadzono jego
ewaluacji, toteż nieznane są jego możliwości. W dalszych pracach zostanie
napisana aplikacja do detekcji wykorzystująca dotychczasowe algorytmy i
implementująca opisane w tym artykule nowe rozwiązanie, po czym zostanie
wykonany szereg testów na stosowanych kodekach dla wszelkiego rodzaju sygnałów
audio.

\printbibliography

\end{document}
